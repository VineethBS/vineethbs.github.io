\textbf{Sending a sequence of bits as frames}
\begin{enumerate}
\item Generate a random sequence of 1000 data bits. 
\item We need to model sending this random sequence of 1000 bits as frames. Each frame consists of 50 data bits and a 13 bit header which is chosen to be a 13 bit Barker sequence. Find out what is a Barker sequence and write a matlab function which can convert a sequence of data bits into frames in the above format.
\item We now need to simulate transmission of the baseband waveform corresponding to the frames that you have generated above.
\item Assume that each bit in a frame is transmitted for a time $T_{b}$. Also assume that consecutive frames are transmitted with a time gap of $D \times T_{b}$ or $D$ bit times, where $D$ is a discrete uniform random variable taking values in $\brac{10,11,12,\cdots,100}$. Assume that there is a random gap of $D$ before the first frame is sent. Note that $D$ is sampled every time before a frame is sent. A bit 1 is transmitted as a pulse of amplitude $A$ and a bit 0 is transmitted as a pulse of amplitude $-A$.
\item Obtain the baseband waveform corresponding to the sequence of frames that you have obtained above. Note that the baseband waveform has to be a sampled representation of the actual continuous time signal.
\end{enumerate}

\myhrule
\noindent
\textbf{Frame synchronization using autocorrelation}
\begin{enumerate}
\item The objective in this subtask is to estimate/obtain the start of each frame from the received waveform.
\item Simulate the reception of the above baseband waveform transmitted over an ideal channel which has an attenuation $\gamma$ (constant for all frequencies) and introduces sampled Gaussian noise of variance $\sigma^{2}$. Please note that the channel should not distort the waveform (use a low pass filter of sufficiently high bandwidth or even an all-pass filter). Please use a matched filter to receive the above baseband waveform.
\item Note that because of the way the transmitted waveform is constructed and the channel (model) over which the waveform is transmitted, sampling the received waveform uniformly at multiples of $T_{b}$ gives you samples from different bit times. However note that during the spaces between the samples (during a sample of $D$, you might be sampling from noise). So obtain a sequence of samples $Z_{n}$ from the received waveform by sampling at multiples of $T_{b}$.
\item Compute the autocorrelation of the sequence $Z_{n}$ with the Barker sequence, what do you observe?
\item Write down a function (in Matlab) which computes the autocorrelation of the Barker sequence with a 13 length moving window of the received sequence. The function should return an autocorrelation value every iteration.
\item Can this function be used to estimate the start of the frames? Devise a decision rule that can be used to detect the start of each frame.
\item Obtain the fraction of frames for which the start time is estimated in error. Call this the frame-start-error-rate.
\item Repeat the above sequence of tasks for a case where 1000 frames are transmitted, i.e., you should start with $1000 \times 50$ bits.
\item Now plot the frame-start-error-rate as a function of $\sigma^{2}$ and write down your observations.
\end{enumerate}

\myhrule
\noindent
\textbf{Early late gating - 1}
\begin{enumerate}
\item We will first simulate a single pulse which is transmitted over a baseband channel arriving at the receiver after a random unknown propagation delay. The received pulse can simply be modelled as a pulse waveform with a random initial shift.
\item Suppose we use a matched filter for receiving and further processing of these pulses. Note that since the random initial shift is not known to the receiver, the receiver samples the output of the matched filter at some time, which it assumes is that time at which the output is maximum.
\item Using Early-Late gating obtain the sampling instant for the case of the single pulse. Note that multiple iterations can be carried out on the single pulse itself to find out this sampling instant.
\end{enumerate}

\myhrule
\noindent
\textbf{Early late gating - 2}
\begin{enumerate}
\item We will consider a slightly different early-late gating method which is applied on a sequence of bits, rather than a single bit as discussed above.
\item Generate a sequence of alternating 0s and 1s of length $N$.
\item Convert the above sequence of bits into a baseband waveform (sampled)
\item Pass the above baseband waveform through a channel with introduces sampled Gaussian noise with variance $\sigma^{2}$, attenuation $\gamma$, and a random delay $D$ (choose $D$ uniformly in the number of samples in $[0, T_{b}]$).
\item Note that when the receiver starts, it does not know the random delay $D$. Therefore it might start sampling the received signal at $T_{b}$ - not accounting for the random delay $D$. 
\item Visualize the received signal after matched filtering using an eye diagram and observe the sample values at the receiver's estimate of the sampling time.
\item Now run the early-late gating method that you have developed in the previous task here, but run it not on the first bit period $T_{b}$, but on multiple bit periods.
\item After $N$ bit times, compute the squared error between the receivers estimate of the sample time and the ideal sample time. Note that this squared error has been obtained for a fixed $\sigma^{2}$ and $\gamma$.
\item Repeat the above sequence of steps for 1000 times (generate different samples of $D$) and obtain the mean of the 1000 squared error quantities.
\item Repeat for different values of $\sigma^{2}$ and plot the mean squared error as a function of $\sigma^{2}$.
\item Repeat for different values of $N$, keeping $\sigma^{2}$ constant and plot the mean squared error as a function of $N$.
\end{enumerate}

\newpage
\myhrule
\noindent
\textbf{Symbol and frame synchronization}
\begin{enumerate}
\item In this task, we will attempt to do symbol and frame synchronization together. We note that in the ``frame synchronization using autocorrelation'' task, symbol synchronization was assumed since the gaps between the frames are multiples of $T_{b}$. Here we will consider the case where the channel introduces a random delay and the gaps between frames are not restricted to be multiples of $T_{b}$.
\item Again generate a random sequence of 1000 data bits. 
\item We need to model sending this random sequence of 1000 bits as frames. Each frame consists of 50 data bits and a 23 bit header which is chosen to be a 10 bit sequence of alternating 1s and 0s and then a 13 bit Barker sequence. Write a matlab function which can convert a sequence of data bits into frames in the above format.
\item We then simulate transmission of the baseband waveform corresponding to the frames that you have generated above.
\item Assume that each bit in a frame is transmitted for a time $T_{b}$. Also assume that consecutive frames are transmitted with a time gap of $D$ samples, where $D$ is a discrete uniform random variable taking values in $\brac{0,1,2,\cdots,50} \times T_{b} \times$ the number of samples in a $T_{b}$. \item Also assume that there is a random gap of $D$ before the first frame is sent. Note that $D$ is sampled every time before a frame is sent. A bit 1 is transmitted as a pulse of amplitude $A$ and a bit 0 is transmitted as a pulse of amplitude $-A$.
\item Obtain the baseband waveform corresponding to the sequence of frames that you have obtained above. Note that the baseband waveform has to be a sampled representation of the actual continuous time signal.
\item Pass the above baseband waveform through a channel with introduces sampled Gaussian noise with variance $\sigma^{2}$, attenuation $\gamma$, and a random delay $D$ (as above).\
\item The received signal is matched filtered, and applied to the early late gating (2) method that you have developed. 
\item The sampling time estimate is used to obtain the samples $Z_{n}$ again. 
\item Now obtain the frame start times using the method that you have used in frame synchronization task.
\item The performance of the synchronization method is evaluated by finding out the average number of frames which are received in error.Call this the frame-start-error-rate.
\item Repeat the above sequence of tasks for a case where 1000 frames are transmitted, i.e., you should start with $1000 \times 50$ bits.
\item Now plot the frame-start-error-rate as a function of $\sigma^{2}$.
\end{enumerate}
