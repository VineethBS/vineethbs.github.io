%%%%%%%%%%%%%%%%%%%%%%%%%%%%%%%%%%%%%%%%%%%%%%%%%%%%%%%%%%%%%%%%
%                                                              %
%  Template for creating lab sheets for AV343                  %
%  Modified from COS511 scribe template by Robert Schapire     %
%                                                              %
%%%%%%%%%%%%%%%%%%%%%%%%%%%%%%%%%%%%%%%%%%%%%%%%%%%%%%%%%%%%%%%%


\documentclass[11pt]{article}
\usepackage{fullpage}
\pagestyle{plain}

\usepackage{amsmath, amsfonts}
\usepackage{fullpage}
\usepackage{listings}

\newcommand{\Exp}{\mathbb{E}}
\newcommand{\Brac}[1]{\bigg\{ #1 \bigg\}}
\newcommand{\Bras}[1]{\bigg[ #1 \bigg]}
\newcommand{\Brap}[1]{\bigg( #1 \bigg)}
\newcommand{\brac}[1]{\left\{ #1 \right\}}
\newcommand{\bras}[1]{\left[ #1 \right]}
\newcommand{\brap}[1]{\left( #1 \right)}

%\setlength{\parindent}{0em}
\newcommand{\myhrule}{
\vspace{0.1in}
\hrule
\vspace{0.1in}
}
\newcommand{\code}{\textbf{Code: }}
\newcommand{\results}{\textbf{Results: }}

\begin{document}

\begin{center}
\bf\large AV343: Communication Systems Lab
\end{center}

\noindent
Lab 1         	 % Enter the lecture number here
\\
Signals and Systems Review % Enter the lab title here
\hfill
Date: 3rd January 2019      % Enter the date of the lab here

\noindent
\myhrule
Vineeth B. S. (Student code)
\myhrule

\medskip

%% Put your answers here
\begin{enumerate}
\item Suppose $x_{1}$ and $x_{2}$ are two finite sequences defined as
  \begin{eqnarray*}
    x_{1}[n] & = & [4,2,6,3,8,1,5] \\
    x_{2}[n] & = & [3,8,6,9,6,7]
  \end{eqnarray*}
  Let the starting index of $x_{1}[n]$ be $-1$ (i.e. $x_{1}[-1] = 4, x_{1}[0] = 2 \dots$) and the starting index of $x_{2}[n]$ be -2.
  Obtain the convolution of $x_{1}[n]$ and $x_{2}[n]$ using Matlab
  \begin{enumerate}
  \item You should first try an implementation on your own without using any inbuilt functions.
  \item Now find out whether there are any inbuilt Matlab functions that can be used to compute convolution and use that.
  \end{enumerate}


  \code
  \begin{lstlisting}
    for i = 1:10
      for j = 1:10
        disp(i + j)
      end
    end
  \end{lstlisting}

  \results
  2
  3
  4
  

\end{enumerate}
\end{document}
