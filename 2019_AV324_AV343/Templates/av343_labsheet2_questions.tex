\begin{enumerate}
\item Let $x(t)$ be a continuous time signal $2cos(2000\pi t)$ defined for all $t$. Let $x[n]$ be a discrete time signal obtained by sampling $x(t)$ with sampling frequency $16000$ Hz for $t \in [0, 8)$ ms (note that $8$ ms is not included).
  \begin{enumerate}
  \item If $N$ is the length of $x[n]$, plot the N-point DFT of $x[n]$.
  \item What is the CTFT of $x(t)$?
  \item We will use $x[n]$'s DFT to think about $x(t)$'s CTFT. Plot the DFT magnitude of $x[n]$. What is the difference between this plot and the CTFT?
  \item Change the frequency axis to the continuous time frequency $\Omega$ domain. Does your DFT plot look like a frequency sampled CTFT? Do you need to change the magnitude axis? What should you scale the magnitude axis with? Why?
  \end{enumerate}
  
\item Let $x(t)$ be a continuous time signal $2cos(2000\pi t) + 5cos(500\pi t)$ defined for all $t$. We define the following sampling rates: $f_{s,1} = 500$, $f_{s,2} = 2000$, $f_{s,3} = 4000$, where all sampling rates have units of Hertz. Let $x_{i}[n]$ be the discrete time signal obtained by sampling $x(t)$ at sampling rate $f_{s, i}$ for $t \in [0, 8)$ ms (note that $8$ ms is not included)
  \begin{enumerate}
  \item What is the CTFT of $x(t)$?
  \item Plot the magnitude of DFTs of each $x_{i}[n]$. Compare the magnitude plots? Are they different? Why do the differences arise?
  \item We note that it is possible to obtain a continuous time signal $x_{i}(t)$ from each $x_{i}[n]$ by using sinc interpolation which we had studied in class. As in question 1, use the DFT of each $x_{i}[n]$ as the frequency sampled CTFT of $x_{i}(t)$, i.e., make appropriate changes to the frequency axis and magnitude axis. What frequency components do you get in each $x_{i}(t)$? How are they different from those in $x(t)$? Why? (Please note that you are not asked to do the interpolation!)
  \end{enumerate}

\item Let $x[n]$ be a sequence defined as $(1,1,1,1)$ for $n = (0, 1, 2, 3)$. Obtain
  \begin{enumerate}
  \item the 8-point DFT of $x[n]$ obtained by padding zeros to $x[n]$ in order to make a 8 length sequence,
  \item the 16-point DFT of $x[n]$ obtained by padding zeros to $x[n]$ in order to make a 16 length sequence,
  \item the 32-point DFT of $x[n]$ obtained by padding zeros to $x[n]$ in order to make a 32 length sequence,
  \item the 1024-point DFT of $x[n]$ obtained by padding zeros to $x[n]$ in order to make a 1024 length sequence.
  \end{enumerate}
  Plot and compare the DFTs that you have obtained above.
  Suppose the above 8, 16, 32, and 1024 length sequences are thought of as being obtained by sampling continous time signals at a sampling rate of 1 Hz, but for different time durations.
  Interpret each of the above DFT magnitude plots as frequency sampled CTFTs of the above continuous time signals.
  What is the significance of the time duration in your interpretation?

\item Let $x(t)$ be $2cos(2000\pi t)$. We define the following sampling rates: $f_{s,1} = 8000$, $f_{s,2} = 16000$, $f_{s,3} = 8888$, $f_{s,4} = 17776$, where all sampling rates have units of Hertz.
  \begin{enumerate}
  \item Let $x_{i}[n]$ be the discrete time signal obtained by sampling $x(t)$ at sampling rate $f_{s,i}$ for $t \in [0, 8)$ ms. 
  \item Let $y_{j}[n]$ be the discrete time signal obtained by sampling $x(t)$ at sampling rate $f_{s,i}$ for $t \in [0, 8]$ ms. 
  \item In each case plot the DFT magnitude and use that to think about the CTFT of $x(t)$.
  \item How do each one of the sampled CTFTs that you have ``computed using DFT'' compare with the actual CTFT? 
  \item Let $z[n]$ be the discrete time sequence obtained by sampling $x(t)$ at $f_{s,1}$ for $[0, 4)$ ms. Extend the length of $z[n]$ using zero padding so that $z[n]$ has the same length as $x_{1}[n]$. Compare the DFT magnitude plot of $z[n]$ with $x_{1}[n]$. Is there a difference? If there is, then explain why? 
  \item If $z(t)$ is the continuous time signal obtained from $z[n]$, what is the relationship between $z(t)$ and $x(t)$?
  \end{enumerate}
\item Use internet resources to find out what dual tone multi-frequency (DTMF) signalling is. For example, look at
  \begin{enumerate}
  \item https://en.wikipedia.org/wiki/Dual-tone\_multi-frequency\_signaling
  \item http://onlinetonegenerator.com/dtmf.html
  \end{enumerate}
  When a key is pressed (e.g., on your landline phone) the tone generator would produce a signal for the duration of the key press which is the sum of a high frequency tone and a low frequency tone.
  The high and low frequencies corresponding to different keys are shown in the following table:
  \begin{center}
    \begin{tabular}{|l|l|l|l|l|}
      1209 Hz & 1336 Hz & 1477 Hz & 1633 Hz & \\
      \hline
      1 & 2 & 3 & A & 697 Hz \\
      4 & 5 & 6 & B & 770 Hz \\
      7 & 8 & 9 & C & 852 Hz \\
      * & 0 & \# & D & 941 Hz \\
      \hline      
    \end{tabular}
  \end{center}
  We will first make a DTMF tone simulator.
  Write a Matlab function ``makeDTMFsignal'' that takes as input
  \begin{enumerate}
  \item The sequence of key's pressed (e.g., this input can be the array [1,2]),
  \item The duration of each key press in seconds (all key presses are assumed to be of the same duration, and there is no time taken between two consecutive key presses),
  \end{enumerate}
  and returns the sampled DTMF signal. Please choose an appropriate sampling rate; justify your choice.
  Note that the sampling rate could be another input into your function or hard-coded in your function.
  Play the sampled DTMF signal through your computer speakers.

  In the second part of this question, you will make a DTMF signal decoder. Note that the DTMF signal can be decoded by looking at its frequency content, which can be computed via the DFT.
  Write a Matlab function ``decodeDTMFsignal'' that takes as input
  \begin{enumerate}
  \item the sampled DTMF signal that you have produced using ``makeDTMFsignal''
  \item the duration of each keypress (note that the duration should be the same as that used for making the sampled DTMF signal you are using as input)
  \item the sampling rate used to make the sampled DTMF signal
  \end{enumerate}
  and returns the sequence of keys that corresponds to the DTMF signal that you have used as input.
  Design the function ``decodeDTMFsignal'' using a flowchart.
  Note that the decoder needs to decide during a key press duration whether two frequencies are present in a signal.
  Since all key presses have the same duration, an approach would be to
  \begin{enumerate}
  \item segment the sampled DTMF signal into smaller duration signals, each duration being that of a keypress duration
  \item apply DFT to find out the frequency content
  \item decide if there is a tone at a particular frequency (how will you make this decision?)
  \end{enumerate}
\end{enumerate}

