\textbf{Baseband communication}
\begin{enumerate}
\item Generation of bits: In digital communications, the source is assumed to produce a stream of bits. The bits are usually modelled as an independent and identically distributed random process with the probability of a bit being $1$ being $0.5$. Generate a random sequence of bits of length $N$ (input) satisfying this property. See generate\_frame\_of\_bits.m for hints to do this.
\item Obtaining the line code: In digital communications, the above bit sequence is then converted to a baseband signal according to the modulation scheme which is used. The baseband signal is obtained by converting $0$ and $1$ into appropriate pulses of duration $T_{b}$. For a bit sequence with $N = 10$, obtain a baseband signal for BASK and BPSK, with $T_{b} = 0.1$ seconds. See generate\_bpsk\_baseband\_signal\_for\_frame.m for hints. Note that all continuous time signals have to be represented by their sampled counterparts.
\item Plot the power spectrum of the baseband signals for BASK and BPSK but with $N =  1000$. (Hint: carefully think about energy vs. power).
\item The rest of the tasks has to be done separately for both BASK and BPSK baseband signals.
\item Simulate sending the baseband signal through an ideal channel with zero propagation delay. See pass\_through\_ideal\_channel.m from channel\_modelling.zip for hints.
\item At the output of the ideal channel, implement a low pass receive filter followed by a matched filter for the baseband pulse shape that you have used. See rx\_frontend\_filter\_baseband.m and matched\_filter\_rectangular.m for hints.
\item Obtain samples from the output of the matched filter; note that since an ideal channel is used, these samples can be taken at multiples of the bit times.
\item Implement a decision making device - a thresholder that will convert the samples to 0 or 1 based on whether your baseband signal is for BASK or BPSK.
\end{enumerate}

\myhrule 
\noindent
\textbf{Passband communication}
\begin{enumerate}
\item Generate a sequence of bits with $N = 10$ as in Task 1 for baseband communication.
\item Generate the baseband signal for the above sequence of bits corresponding to both BASK and BPSK as in Task 2 for baseband communication.
\item The rest of the tasks have to be done for both BASK and BPSK if not explicitly mentioned.
\item Plot the power spectrum of the baseband signal.
\item Modulate the baseband signal using a carrier of frequency $50$Hz. See modulate\_baseband\_signal.m for hints.
\item Plot the power spectrum of the modulated signal.
\item Simulate sending the modulated signal through an ideal passband channel with zero propagation delay.
\item At the output of a channel implement a passband receive filter with a passband corresponding to the bandwidth of the signal that you are sending through the channel. See rx\_frontend\_filter\_passband.m.
\item Implement a coherent demodulator for both BASK and BPSK. Plot the spectrum of the output of the coherent demodulator for both cases. See demodulate\_coherent\_bpsk.m.
\item At the output of the demodulator, implement a matched filter for the pulse shape that you have used.
\item Obtain samples from the output of the matched filter; note that since an ideal channel is used, these samples can be taken at multiples of the bit times.
\item Implement a decision making device - a thresholder that will convert the samples to 0 or 1 based on whether your baseband signal is for BASK or BPSK.
\item (Optional): Implement a non-coherent demodulator for BASK.
\end{enumerate}
