\textbf{Intersymbol interference - baseband channels}
\begin{enumerate}
\item We will study and visualize the effect of intersymbol interference (ISI) using the baseband channel models developed in Labsheet 6 (Chapter \ref{chapter:channel_models} in the lab manual). 
\item Consider the baseband channel modelled as a low pass filter. We will investigate what happens to a baseband waveform as it passes through a low pass or baseband channel.
\item We will use a sampling frequency of $100$ Hz for this task.
\item Use a low pass filter with a passband edge of $10$ Hz with a passband gain of $1$ to model the channel. Obtain and plot the magnitude spectrum of the channel.
\item Generate a baseband BPSK signal $b(t)$ corresponding to a random sequence of 100 bits for $T_{b} = 0.1s$.
\item Plot the eye diagram of this baseband BPSK signal. You should observe that this is the eye diagram of a signal without ISI.
\item Pass the baseband BPSK signal through the channel model and plot the eye diagram of the channel output. Comment on what you have observed.
\item Plot the eye diagrams for $T_{b} = 0.05$ and $T_{b} = 0.2$. Comment on your observations.
\item Suppose the channel output is passed through a matched filter for the case of $T_{b} = 0.1$. Plot the eye diagram of the matched-filtered received signal. What differences do you observe?
\end{enumerate}

\myhrule
\noindent
\textbf{Intersymbol interference - passband channels}
\begin{enumerate}
\item Consider the passband channel modelled as a band-pass filter. We will investigate what happens to a baseband waveform as it is modulated, passed through the passband channel, and demodulated.
\item We will use a sampling frequency of $1 kHz$ for this task.
\item Use a bandpass filter with a center frequency of $100 Hz$, with a one-sided bandwidth of $20 Hz$ to model the passband channel. Obtain and plot the magnitude response of the channel.
\item Generate a baseband BPSK signal as in the section above and modulate it and pass through the channel. Demodulate the channel output and plot the eye diagram of the demodulated channel output.
\item Comment on whether there are any fundamental differences between nature of ISI for passband and baseband communication.
\end{enumerate}

\myhrule
\noindent
\textbf{Pulse shaping}
\begin{enumerate}
\item Since we consider ISI at baseband, we will only look at baseband channels for the rest of this labsheet.
\item We will use a sampling frequency of $100$ Hz for this task.
\item Generate a baseband BPSK signal $b_{r}(t)$ corresponding to a random sequence of 1000 bits for $T_{b} = 0.1s$. Note that this baseband BPSK signal should be generated using the rectangular pulse shape.
\item Generate a baseband BPSK signal $b_{s}(t)$ corresponding to the same random sequence of bits used above, but using the sinc pulse shape. 
  \[AT_{b} \frac{sin\big(\pi \frac{t}{T_{b}}\big)}{t}\]
Note that the sinc pulse shape needs to be truncated to duration of $5T_{b}$ so that the truncated pulse shape is symmetric.

\item Generate a baseband BPSK signal $b_{c}(t)$ corresponding to the same random sequence of bits used above, but using the raised cosine pulse shape. 
  \[ AT_{b} \frac{sin\big(\pi \frac{t}{T_{b}}\big)}{t} \frac{cos\big(\pi \alpha \frac{t}{T_{b}}\big)}{\big(1 - \big(2\alpha \frac{t}{Tb}\big)^{2}\big)}.\]
Note that the raised cosine pulse shape needs to be truncated to duration of $5T_{b}$ so that the truncated pulse shape is symmetric.

\item Plot the power spectral densities of the signals $b_{r}(t), b_{s}(t)$, and $b_{c}(t)$. Comment on the differences in the three spectra.
\item Plot the eye diagrams of $b_{r}(t), b_{s}(t)$, and $b_{c}(t)$. Comment on the differences and main features of each eye diagram.
\item Plot the spectra and eye diagrams of $b_{c}(t)$ for different values of $\alpha$. What do you observe?

\item Generate a baseband BPSK signal $b_{rc}(t)$ using the square root raised cosine pulse shape. Obtain the power spectral density and eye diagram for this pulse shape.

\item For each of the signals $b_{r}(t), b_{s}(t), b_{c}(t)$, and $b_{rc}(t)$ obtain the channel output when the respective baseband signals are sent through a baseband channel. Use a low pass filter with a passband edge of $10$ Hz with a passband gain of $1$ to model the channel.

\item Obtain the PSD and eye diagram of the channel output in each of the cases. Compare the bandwidth and eye width properties for each of the pulse shapes for $T_{b} = 0.1, 0.05$ and $0.2$.

\end{enumerate}
