\begin{enumerate}
\item In class review of how RTL-SDR works.
\begin{itemize}
    \item What is the output from an RTL-SDR?
    \item Review of complex baseband, passband sampling and IQ data.
    \item A good read - https://www.dsprelated.com/showarticle/192.php
    \item Another one - http://whiteboard.ping.se/SDR/IQ
\end{itemize}
\item In this lab, you will continue your exploration of GNU Radio and its use with a specific software defined radio receiver called RTL-SDR.
    \begin{itemize}
    \item It is essential that you have finished the last lab, installed GNU Radio with support for RTL-SDR
    \item Every batch should get an RTL-SDR from the lab staff/instructor which should be returned at the end of the lab in proper condition, along with any additional equipment (e.g., antennae)
    \item You will also need to download FM-Demonstration.zip file containing FM\_0.grc, $\dots$, FM\_6.grc for this lab.
    \end{itemize}
\item Your first task is to listen to any FM station using the FM\_0.grc file. Attach the RTL-SDR device to your laptop and use FM\_0.grc. How will you change the channel/station that you are listening to? How many channels can you listen to? You should report the channel frequency value as well as perceived quality (understandable, gibberish etc.) in a table.
\item Make a signal flow diagram corresponding to the flowgraph in FM\_0.grc and write down what each of the blocks do.
\item You should use FM\_1.grc for this task. Open and run the flowgraph. What do you observe? Make a signal flow diagram corresponding to the flowgraph in FM\_1.grc and write down what each of the blocks do.
\item You should use FM\_5.grc for this task. Open and run the flowgraph.What do you observe? Make a signal flow diagram corresponding to the flowgraph in FM\_5.grc and write down what each of the blocks do.
\item You should use FM\_6.grc for this task. Open and run the flowgraph.What do you observe? Make a signal flow diagram corresponding to the flowgraph in FM\_6.grc and write down what each of the blocks do.
\item In the above tasks, your observations should be backed with plots of spectra of signals which you think are important to your observations.
\end{enumerate}
